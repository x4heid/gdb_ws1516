\documentclass[ngerman]{gdb-aufgabenblatt}


\renewcommand{\Aufgabenblatt}{6}
\renewcommand{\Ausgabedatum}{Mi. 06.01.2016}
\renewcommand{\Abgabedatum}{Fr. 22.01.2016}
\renewcommand{\Gruppe}{Kraemer, Mirzada, Frangopoulos, Heid}
\renewcommand{\STiNEGruppe}{06}
\renewcommand{\Semester}{WS 2015/16}


\begin{document}

\newcommand{\vertexnode}[6]{
	\node[vertex] (#1)[#2]{\nodepart{one} #3
					 	   \nodepart{two} #4
					 	   \nodepart{three} #5
					 	   \nodepart{four} #6};
}

\newcommand{\vertexnodeShort}[4]{
	\node[vertex2] (#1) [#2]{\nodepart{one} #3
							 \nodepart{two} #4};
}

\section*{Aufgabe 1: B-B�ume}
\subsection*{a) Einf�gen von Datens�tzen}
\textbf{Ursprungsbaum}\\
\begin{tikzpicture}[node distance= 1.5cm and 1cm]
\tikzstyle{vertex} = [rectangle split, rectangle split horizontal, rectangle split parts=4, draw, text width = 0.3cm, align = center, minimum height=0.5cm, minimum width=2cm]
\node[vertex] (root){\nodepart{one} 30
					 \nodepart{two} 83
					 \nodepart{three} 
					 \nodepart{four} };
					 
\vertexnode{child1}{below= of root}{38}{72}{75}{}
\vertexnode{child2}{right= of child1}{86}{98}{}{}
\vertexnode{child3}{left= of child1}{11}{27}{}{}

\draw[->, thick] (root.205) -- (child1);
\draw[->, thick] (root.south) -- (child2);
\draw[->, thick] (root.south west) -- (child3);
\end{tikzpicture}\\

\textbf{79, 85: Einfaches Einf�gen}\\
\begin{tikzpicture}[node distance= 1.5cm and 1cm]
\tikzstyle{vertex} = [rectangle split, rectangle split horizontal, rectangle split parts=4, draw, text width = 0.3cm, align = center, minimum height=0.5cm, minimum width=2cm]
\node[vertex] (root){\nodepart{one} 30
					 \nodepart{two} 83
					 \nodepart{three} 
					 \nodepart{four} };
					 
\vertexnode{child1}{below= of root}{38}{72}{75}{79}
\vertexnode{child2}{right= of child1}{85}{86}{98}{}
\vertexnode{child3}{left= of child1}{11}{27}{}{}

\draw[->, thick] (root.205) -- (child1);
\draw[->, thick] (root.south) -- (child2);
\draw[->, thick] (root.south west) -- (child3);
\end{tikzpicture}

\textbf{81: Einf�gen mit Splitten}\\
\begin{tikzpicture}[node distance= 1.5cm and 1cm]
\tikzstyle{vertex} = [rectangle split, rectangle split horizontal, rectangle split parts=4, draw, text width = 0.3cm, align = center, minimum height=0.5cm, minimum width=2cm]
\node[vertex] (root){\nodepart{one} 30
					 \nodepart{two} 75
					 \nodepart{three} 83
					 \nodepart{four} };
					 
\vertexnode{child1}{below= of root}{38}{72}{}{}
\vertexnode{child2}{right= of child1}{79}{81}{}{}
\vertexnode{child3}{left= of child1}{11}{27}{}{}
\vertexnode{child4}{right= of child2}{85}{86}{98}{}

\draw[->, thick] (root.205) -- (child1);
\draw[->, thick] (root.south) -- (child2);
\draw[->, thick] (root.south west) -- (child3);
\draw[->, thick] (root.335) -- (child4);
\end{tikzpicture}

\textbf{1, 29, 92: Einfaches Einf�gen}\\
\begin{tikzpicture}[node distance= 1.5cm and 1cm]
\tikzstyle{vertex} = [rectangle split, rectangle split horizontal, rectangle split parts=4, draw, text width = 0.3cm, align = center, minimum height=0.5cm, minimum width=2cm]
\node[vertex] (root){\nodepart{one} 30
					 \nodepart{two} 75
					 \nodepart{three} 83
					 \nodepart{four} };
					 
\vertexnode{child1}{below= of root}{38}{72}{}{}
\vertexnode{child2}{right= of child1}{79}{81}{}{}
\vertexnode{child3}{left= of child1}{1}{11}{27}{29}
\vertexnode{child4}{right= of child2}{85}{86}{92}{98}

\draw[->, thick] (root.205) -- (child1);
\draw[->, thick] (root.south) -- (child2);
\draw[->, thick] (root.south west) -- (child3);
\draw[->, thick] (root.335) -- (child4);
\end{tikzpicture}

\textbf{9: Einf�gen mit Splitten}\\
\begin{tikzpicture}[node distance= 1.5cm and 1cm]
\tikzstyle{vertex} = [rectangle split, rectangle split horizontal, rectangle split parts=4, draw, text width = 0.3cm, align = center, minimum height=0.5cm, minimum width=2cm]
\node[vertex] (root){\nodepart{one} 11
					 \nodepart{two} 30
					 \nodepart{three} 75
					 \nodepart{four} 83};
					 
\vertexnode{child1}{below = of root}{38}{72}{}{}
\vertexnode{child2}{below left= of root}{27}{29}{}{}
\vertexnode{child3}{left= of child2}{1}{9}{}{}
\vertexnode{child4}{below right= of root}{79}{81}{}{}
\vertexnode{child5}{right= of child4}{85}{86}{92}{98}

\draw[->, thick] (root.south west) -- (child3);
\draw[->, thick] (root.205) -- (child2);
\draw[->, thick] (root.south) -- (child1);
\draw[->, thick] (root.335) -- (child4);
\draw[->, thick] (root.south east) -- (child5);
\end{tikzpicture}

\textbf{94: Einf�gen mit Splitten}\\
\begin{tikzpicture}[node distance= 1.5cm and 1cm]
\tikzstyle{vertex} = [rectangle split, rectangle split horizontal, rectangle split parts=4, draw, text width = 0.3cm, align = center, minimum height=0.5cm, minimum width=2cm]
\node[vertex] (root){\nodepart{one} 75
					 \nodepart{two} 
					 \nodepart{three} 
					 \nodepart{four}  };
					 
%\vertexnode{child1}{below = of root}{38}{72}{}{}
\vertexnode{child1}{below left= 1.5cm and 2.5cm of root}{11}{30}{}{}
%\vertexnode{child3}{left= of child2}{1}{9}{}{}
\vertexnode{child2}{below right= 1.5cm and 2.5cm of root}{83}{92}{}{}
%\vertexnode{child5}{right= of child4}{85}{86}{92}{98}

\vertexnode{gchild1}{below left= of child1}{1}{9}{}{}
\vertexnode{gchild2}{below = of child1}{27}{29}{}{}
\vertexnode{gchild3}{below right= of child1}{38}{72}{}{}
\vertexnode{gchild4}{below left= of child2}{79}{81}{}{}
\vertexnode{gchild5}{below = of child2}{85}{86}{}{}
\vertexnode{gchild6}{below right= of child2}{94}{98}{}{}


\draw[->, thick] (root.south west) -- (child1);
\draw[->, thick] (root.205) -- (child2);

\draw[->, thick] (child1.south west) -- (gchild1);
\draw[->, thick] (child1.205) -- (gchild2);
\draw[->, thick] (child1.south) -- (gchild3);
\draw[->, thick] (child2.south west) -- (gchild4);
\draw[->, thick] (child2.205) -- (gchild5);
\draw[->, thick] (child2.south) -- (gchild6);

\end{tikzpicture}

\textbf{16, 31, 21: Einfaches Einf�gen}\\
\begin{tikzpicture}[node distance= 1.5cm and 1cm]
\tikzstyle{vertex} = [rectangle split, rectangle split horizontal, rectangle split parts=4, draw, text width = 0.3cm, align = center, minimum height=0.5cm, minimum width=2cm]
\node[vertex] (root){\nodepart{one} 75
					 \nodepart{two} 
					 \nodepart{three} 
					 \nodepart{four}  };
					 
%\vertexnode{child1}{below = of root}{38}{72}{}{}
\vertexnode{child1}{below left= 1.5cm and 2.5cm of root}{11}{30}{}{}
%\vertexnode{child3}{left= of child2}{1}{9}{}{}
\vertexnode{child2}{below right= 1.5cm and 2.5cm of root}{83}{92}{}{}
%\vertexnode{child5}{right= of child4}{85}{86}{92}{98}

\vertexnode{gchild1}{below left= of child1}{1}{9}{}{}
\vertexnode{gchild2}{below = of child1}{16}{21}{27}{29}
\vertexnode{gchild3}{below right= of child1}{31}{38}{72}{}
\vertexnode{gchild4}{below left= of child2}{79}{81}{}{}
\vertexnode{gchild5}{below = of child2}{85}{86}{}{}
\vertexnode{gchild6}{below right= of child2}{94}{98}{}{}


\draw[->, thick] (root.south west) -- (child1);
\draw[->, thick] (root.205) -- (child2);

\draw[->, thick] (child1.south west) -- (gchild1);
\draw[->, thick] (child1.205) -- (gchild2);
\draw[->, thick] (child1.south) -- (gchild3);
\draw[->, thick] (child2.south west) -- (gchild4);
\draw[->, thick] (child2.205) -- (gchild5);
\draw[->, thick] (child2.south) -- (gchild6);

\end{tikzpicture}

\textbf{14: Einf�gen mit Splitten}\\
\begin{tikzpicture}[node distance= 1.5cm and 1cm]
\tikzstyle{vertex} = [rectangle split, rectangle split horizontal, rectangle split parts=4, draw, text width = 0.3cm, align = center, minimum height=0.5cm, minimum width=2cm]
\node[vertex] (root){\nodepart{one} 75
					 \nodepart{two} 
					 \nodepart{three} 
					 \nodepart{four}  };
					 
%\vertexnode{child1}{below = of root}{38}{72}{}{}
\vertexnode{child1}{below left= 1.5cm and 2.5cm of root}{11}{21}{30}{}
%\vertexnode{child3}{left= of child2}{1}{9}{}{}
\vertexnode{child2}{below right= 1.5cm and 2.5cm of root}{83}{92}{}{}
%\vertexnode{child5}{right= of child4}{85}{86}{92}{98}

\vertexnode{gchild1}{below left= of child1}{1}{9}{}{}
\vertexnode{gchild2}{below = of child1}{14}{16}{}{}
\vertexnode{gchild3}{below right= of child1}{31}{38}{72}{}
\vertexnode{gchild7}{below right= 0.5cm and 0.0cm of gchild2}{27}{29}{}{}
\vertexnode{gchild4}{below left= of child2}{79}{81}{}{}
\vertexnode{gchild5}{below = of child2}{85}{86}{}{}
\vertexnode{gchild6}{below right= of child2}{94}{98}{}{}


\draw[->, thick] (root.south west) -- (child1);
\draw[->, thick] (root.205) -- (child2);

\draw[->, thick] (child1.south west) -- (gchild1);
\draw[->, thick] (child1.205) -- (gchild2);
\draw[->, thick] (child1.south) -- (gchild7);
\draw[->, thick] (child1.335) -- (gchild3);
\draw[->, thick] (child2.south west) -- (gchild4);
\draw[->, thick] (child2.205) -- (gchild5);
\draw[->, thick] (child2.south) -- (gchild6);

\end{tikzpicture}




\newpage
\subsection*{b) L�schen von Datens�tzen}
\textbf{Ursprungsbaum}\\
\begin{tikzpicture}[node distance= 1.5cm and 1cm]
\tikzstyle{vertex} = [rectangle split, rectangle split horizontal, rectangle split parts=4, draw, text width = 0.3cm, align = center, minimum height=0.5cm, minimum width=2cm]
\node[vertex] (root){\nodepart{one} 43
					 \nodepart{two} 
					 \nodepart{three} 
					 \nodepart{four} };
					 
\vertexnode{child1}{below left= 1.5cm and 2.5cm of root}{21}{32}{}{}
\vertexnode{child2}{below right= 1.5cm and 2.5cm of root}{58}{94}{}{}

\vertexnode{gchild1}{below left= of child1}{4}{8}{}{}
\vertexnode{gchild2}{below = of child1}{24}{28}{}{}
\vertexnode{gchild3}{below right= of child1}{34}{38}{41}{}
\vertexnode{gchild4}{below left= of child2}{45}{52}{53}{57}
\vertexnode{gchild5}{below = of child2}{61}{82}{}{}
\vertexnode{gchild6}{below right= of child2}{97}{98}{}{}



\draw[->, thick] (root.south west) -- (child1);
\draw[->, thick] (root.205) -- (child2);

\draw[->, thick] (child1.south west) -- (gchild1);
\draw[->, thick] (child1.205) -- (gchild2);
\draw[->, thick] (child1.south) -- (gchild3);
\draw[->, thick] (child2.south west) -- (gchild4);
\draw[->, thick] (child2.205) -- (gchild5);
\draw[->, thick] (child2.south) -- (gchild6);
\end{tikzpicture}\\

\textbf{52, 34: Einfaches L�schen}\\
\begin{tikzpicture}[node distance= 1.5cm and 1cm]
\tikzstyle{vertex} = [rectangle split, rectangle split horizontal, rectangle split parts=4, draw, text width = 0.3cm, align = center, minimum height=0.5cm, minimum width=2cm]
\node[vertex] (root){\nodepart{one} 43
					 \nodepart{two} 
					 \nodepart{three} 
					 \nodepart{four} };
					 
\vertexnode{child1}{below left= 1.5cm and 2.5cm of root}{21}{32}{}{}
\vertexnode{child2}{below right= 1.5cm and 2.5cm of root}{58}{94}{}{}

\vertexnode{gchild1}{below left= of child1}{4}{8}{}{}
\vertexnode{gchild2}{below = of child1}{24}{28}{}{}
\vertexnode{gchild3}{below right= of child1}{38}{41}{}{}
\vertexnode{gchild4}{below left= of child2}{45}{53}{57}{}
\vertexnode{gchild5}{below = of child2}{61}{82}{}{}
\vertexnode{gchild6}{below right= of child2}{97}{98}{}{}



\draw[->, thick] (root.south west) -- (child1);
\draw[->, thick] (root.205) -- (child2);

\draw[->, thick] (child1.south west) -- (gchild1);
\draw[->, thick] (child1.205) -- (gchild2);
\draw[->, thick] (child1.south) -- (gchild3);
\draw[->, thick] (child2.south west) -- (gchild4);
\draw[->, thick] (child2.205) -- (gchild5);
\draw[->, thick] (child2.south) -- (gchild6);
\end{tikzpicture}\\

\textbf{97: L�schen mit Mischen}\\
\begin{tikzpicture}[node distance= 1.5cm and 1cm]
\tikzstyle{vertex} = [rectangle split, rectangle split horizontal, rectangle split parts=4, draw, text width = 0.3cm, align = center, minimum height=0.5cm, minimum width=2cm]
\node[vertex] (root){\nodepart{one} 28
					 \nodepart{two} 43
					 \nodepart{three} 58
					 \nodepart{four} };

\vertexnode{child2}{below left= of root}{32}{38}{41}{}				 
\vertexnode{child1}{left= of child2}{4}{8}{21}{24}

\vertexnode{child3}{below right= of root}{45}{53}{57}{}
\vertexnode{child4}{right= of child3}{61}{82}{94}{98}

%\vertexnode{gchild1}{below left= of child1}{4}{8}{}{}
%\vertexnode{gchild2}{below = of child1}{24}{28}{}{}
%\vertexnode{gchild3}{below right= of child1}{38}{41}{}{}
%\vertexnode{gchild4}{below left= of child2}{45}{53}{57}{}
%\vertexnode{gchild5}{below = of child2}{61}{82}{}{}
%\vertexnode{gchild6}{below right= of child2}{97}{98}{}{}



\draw[->, thick] (root.south west) -- (child1);
\draw[->, thick] (root.205) -- (child2);
\draw[->, thick] (root.south) -- (child3);
\draw[->, thick] (root.335) -- (child4);


%\draw[->, thick] (child1.south west) -- (gchild1);
%\draw[->, thick] (child1.205) -- (gchild2);
%\draw[->, thick] (child1.south) -- (gchild3);
%\draw[->, thick] (child2.south west) -- (gchild4);
%\draw[->, thick] (child2.205) -- (gchild5);
%\draw[->, thick] (child2.south) -- (gchild6);
\end{tikzpicture}\\

\textbf{41, 94: Einfaches L�schen}\\
\begin{tikzpicture}[node distance= 1.5cm and 1cm]
\tikzstyle{vertex} = [rectangle split, rectangle split horizontal, rectangle split parts=4, draw, text width = 0.3cm, align = center, minimum height=0.5cm, minimum width=2cm]
\node[vertex] (root){\nodepart{one} 28
					 \nodepart{two} 43
					 \nodepart{three} 58
					 \nodepart{four} };

\vertexnode{child2}{below left= of root}{32}{38}{}{}				 
\vertexnode{child1}{left= of child2}{4}{8}{21}{24}

\vertexnode{child3}{below right= of root}{45}{53}{57}{}
\vertexnode{child4}{right= of child3}{61}{82}{98}{}

%\vertexnode{gchild1}{below left= of child1}{4}{8}{}{}
%\vertexnode{gchild2}{below = of child1}{24}{28}{}{}
%\vertexnode{gchild3}{below right= of child1}{38}{41}{}{}
%\vertexnode{gchild4}{below left= of child2}{45}{53}{57}{}
%\vertexnode{gchild5}{below = of child2}{61}{82}{}{}
%\vertexnode{gchild6}{below right= of child2}{97}{98}{}{}



\draw[->, thick] (root.south west) -- (child1);
\draw[->, thick] (root.205) -- (child2);
\draw[->, thick] (root.south) -- (child3);
\draw[->, thick] (root.335) -- (child4);


%\draw[->, thick] (child1.south west) -- (gchild1);
%\draw[->, thick] (child1.205) -- (gchild2);
%\draw[->, thick] (child1.south) -- (gchild3);
%\draw[->, thick] (child2.south west) -- (gchild4);
%\draw[->, thick] (child2.205) -- (gchild5);
%\draw[->, thick] (child2.south) -- (gchild6);
\end{tikzpicture}\\

\textbf{28: L�schen mit Ausgleichen}\\
\begin{tikzpicture}[node distance= 1.5cm and 1cm]
\tikzstyle{vertex} = [rectangle split, rectangle split horizontal, rectangle split parts=4, draw, text width = 0.3cm, align = center, minimum height=0.5cm, minimum width=2cm]
\node[vertex] (root){\nodepart{one} 24
					 \nodepart{two} 43
					 \nodepart{three} 58
					 \nodepart{four} };

\vertexnode{child2}{below left= of root}{32}{38}{}{}				 
\vertexnode{child1}{left= of child2}{4}{8}{21}{}

\vertexnode{child3}{below right= of root}{45}{53}{57}{}
\vertexnode{child4}{right= of child3}{61}{82}{98}{}

%\vertexnode{gchild1}{below left= of child1}{4}{8}{}{}
%\vertexnode{gchild2}{below = of child1}{24}{28}{}{}
%\vertexnode{gchild3}{below right= of child1}{38}{41}{}{}
%\vertexnode{gchild4}{below left= of child2}{45}{53}{57}{}
%\vertexnode{gchild5}{below = of child2}{61}{82}{}{}
%\vertexnode{gchild6}{below right= of child2}{97}{98}{}{}



\draw[->, thick] (root.south west) -- (child1);
\draw[->, thick] (root.205) -- (child2);
\draw[->, thick] (root.south) -- (child3);
\draw[->, thick] (root.335) -- (child4);


%\draw[->, thick] (child1.south west) -- (gchild1);
%\draw[->, thick] (child1.205) -- (gchild2);
%\draw[->, thick] (child1.south) -- (gchild3);
%\draw[->, thick] (child2.south west) -- (gchild4);
%\draw[->, thick] (child2.205) -- (gchild5);
%\draw[->, thick] (child2.south) -- (gchild6);
\end{tikzpicture}\\

\textbf{98, 4: Einfaches L�schen}\\
\begin{tikzpicture}[node distance= 1.5cm and 1cm]
\tikzstyle{vertex} = [rectangle split, rectangle split horizontal, rectangle split parts=4, draw, text width = 0.3cm, align = center, minimum height=0.5cm, minimum width=2cm]
\node[vertex] (root){\nodepart{one} 24
					 \nodepart{two} 43
					 \nodepart{three} 58
					 \nodepart{four} };

\vertexnode{child2}{below left= of root}{32}{38}{}{}				 
\vertexnode{child1}{left= of child2}{8}{21}{}{}

\vertexnode{child3}{below right= of root}{45}{53}{57}{}
\vertexnode{child4}{right= of child3}{61}{82}{}{}

%\vertexnode{gchild1}{below left= of child1}{4}{8}{}{}
%\vertexnode{gchild2}{below = of child1}{24}{28}{}{}
%\vertexnode{gchild3}{below right= of child1}{38}{41}{}{}
%\vertexnode{gchild4}{below left= of child2}{45}{53}{57}{}
%\vertexnode{gchild5}{below = of child2}{61}{82}{}{}
%\vertexnode{gchild6}{below right= of child2}{97}{98}{}{}



\draw[->, thick] (root.south west) -- (child1);
\draw[->, thick] (root.205) -- (child2);
\draw[->, thick] (root.south) -- (child3);
\draw[->, thick] (root.335) -- (child4);


%\draw[->, thick] (child1.south west) -- (gchild1);
%\draw[->, thick] (child1.205) -- (gchild2);
%\draw[->, thick] (child1.south) -- (gchild3);
%\draw[->, thick] (child2.south west) -- (gchild4);
%\draw[->, thick] (child2.205) -- (gchild5);
%\draw[->, thick] (child2.south) -- (gchild6);
\end{tikzpicture}\\





\newpage
\section*{Aufgabe 2: B*-B�ume}
\subsection*{a) Einf�gen von Datens�tzen}
\textbf{Ursprungsbaum}\\
\begin{tikzpicture}[node distance= 1cm and 1cm]
\tikzstyle{vertex} = [rectangle split, rectangle split horizontal, rectangle split parts=4, draw, text width = 0.3cm, align = center, minimum height=0.5cm, minimum width=2cm]
\tikzstyle{vertex2} = [rectangle split, rectangle split horizontal, rectangle split parts=2, draw, text width = 0.3cm, align = center, minimum height=0.5cm, minimum width=1cm]
\node[vertex2] (root){\nodepart{one} 9
					 \nodepart{two}  };
\vertexnode{child1}{below left= of root}{1}{5}{7}{8}
\vertexnode{child2}{below right= of root}{40}{53}{61}{}

\draw[->, very thick] (root.south west) -- (child1);
\draw[->, very thick] (root.south) -- (child2);

\end{tikzpicture}\\

\textbf{64: Einfaches Einf�gen}\\
\begin{tikzpicture}[node distance= 1cm and 1cm]
\tikzstyle{vertex} = [rectangle split, rectangle split horizontal, rectangle split parts=4, draw, text width = 0.3cm, align = center, minimum height=0.5cm, minimum width=2cm]
\tikzstyle{vertex2} = [rectangle split, rectangle split horizontal, rectangle split parts=2, draw, text width = 0.3cm, align = center, minimum height=0.5cm, minimum width=1cm]
\node[vertex2] (root){\nodepart{one} 9
					 \nodepart{two}  };
\vertexnode{child1}{below left= of root}{1}{5}{7}{8}
\vertexnode{child2}{below right= of root}{40}{53}{61}{64}

\draw[->, very thick] (root.south west) -- (child1);
\draw[->, very thick] (root.south) -- (child2);

\end{tikzpicture}\\

\textbf{3: Einf�gen mit Splitten}\\
\begin{tikzpicture}[node distance= 1cm and 1cm]
\tikzstyle{vertex} = [rectangle split, rectangle split horizontal, rectangle split parts=4, draw, text width = 0.3cm, align = center, minimum height=0.5cm, minimum width=2cm]
\tikzstyle{vertex2} = [rectangle split, rectangle split horizontal, rectangle split parts=2, draw, text width = 0.3cm, align = center, minimum height=0.5cm, minimum width=1cm]
\node[vertex2] (root){\nodepart{one} 5
					 \nodepart{two}  9};
\vertexnode{child1}{below left= of root}{1}{3}{}{}
\vertexnode{child2}{below right= of root}{40}{53}{61}{64}
\vertexnode{child3}{below = of root}{7}{8}{}{}

\draw[->, very thick] (root.south west) -- (child1);
\draw[->, very thick] (root.south east) -- (child2);
\draw[->, very thick] (root.south) -- (child3);

\end{tikzpicture}\\

\textbf{6: Einfaches Einf�gen}\\
\begin{tikzpicture}[node distance= 1cm and 1cm]
\tikzstyle{vertex} = [rectangle split, rectangle split horizontal, rectangle split parts=4, draw, text width = 0.3cm, align = center, minimum height=0.5cm, minimum width=2cm]
\tikzstyle{vertex2} = [rectangle split, rectangle split horizontal, rectangle split parts=2, draw, text width = 0.3cm, align = center, minimum height=0.5cm, minimum width=1cm]
\node[vertex2] (root){\nodepart{one} 5
					 \nodepart{two}  9};
\vertexnode{child1}{below left= of root}{1}{3}{}{}
\vertexnode{child2}{below right= of root}{40}{53}{61}{64}
\vertexnode{child3}{below = of root}{6}{7}{8}{}

\draw[->, very thick] (root.south west) -- (child1);
\draw[->, very thick] (root.south east) -- (child2);
\draw[->, very thick] (root.south) -- (child3);

\end{tikzpicture}\\

\textbf{80: Einf�gen mit Splitten}\\
\begin{tikzpicture}[node distance= 1cm and 1cm]
\tikzstyle{vertex} = [rectangle split, rectangle split horizontal, rectangle split parts=4, draw, text width = 0.3cm, align = center, minimum height=0.5cm, minimum width=2cm]
\tikzstyle{vertex2} = [rectangle split, rectangle split horizontal, rectangle split parts=2, draw, text width = 0.3cm, align = center, minimum height=0.5cm, minimum width=1cm]
\node[vertex2] (root){\nodepart{one} 5
					 \nodepart{two}  9};
\vertexnode{child1}{below left= of root}{1}{3}{}{}
\vertexnode{child2}{below right= of root}{40}{53}{61}{64}
\vertexnode{child3}{below = of root}{6}{7}{8}{}

\draw[->, very thick] (root.south west) -- (child1);
\draw[->, very thick] (root.south east) -- (child2);
\draw[->, very thick] (root.south) -- (child3);

\end{tikzpicture}\\


\subsection*{b) L�schen von Datens�tzen}
\textbf{Ursprungsbaum}\\
\begin{tikzpicture}[node distance= 1cm and 1cm]
\tikzstyle{vertex} = [rectangle split, rectangle split horizontal, rectangle split parts=4, draw, text width = 0.3cm, align = center, minimum height=0.5cm, minimum width=2cm]
\tikzstyle{vertex2} = [rectangle split, rectangle split horizontal, rectangle split parts=2, draw, text width = 0.3cm, align = center, minimum height=0.5cm, minimum width=1cm]
\node[vertex2] (root){\nodepart{one} 54
					 \nodepart{two}  };
\vertexnodeShort{child1}{below left=1cm and 2cm of root}{34}{40}
\vertexnodeShort{child2}{below right=1cm and 2cm of root}{76}{}

\vertexnodeShort{gchild1}{below left=of child1}{12}{14}
\vertexnodeShort{gchild2}{below =of child1}{38}{}
\vertexnodeShort{gchild3}{below right=of child1}{44}{46}
\vertexnodeShort{gchild4}{below left=of child2}{76}{}
\vertexnodeShort{gchild5}{below =of child2}{68}{}
\vertexnodeShort{gchild6}{below right=of child2}{86}{}

\draw[->, very thick] (root.south west) -- (child1);
\draw[->, very thick] (root.south) -- (child2);

\draw[->, very thick] (child1.south west) -- (gchild1);
\draw[->, very thick] (child1.south) -- (gchild2);
\draw[->, very thick] (child1.south east) -- (gchild3);

\draw[->, very thick] (child2.south west) -- (gchild4);
\draw[->, very thick] (child2.south) -- (gchild5);
\draw[->, very thick] (child2.south east) -- (gchild6);
\end{tikzpicture}

\textbf{14: Einfaches L�schen}\\
\begin{tikzpicture}[node distance= 1cm and 1cm]
\tikzstyle{vertex} = [rectangle split, rectangle split horizontal, rectangle split parts=4, draw, text width = 0.3cm, align = center, minimum height=0.5cm, minimum width=2cm]
\tikzstyle{vertex2} = [rectangle split, rectangle split horizontal, rectangle split parts=2, draw, text width = 0.3cm, align = center, minimum height=0.5cm, minimum width=1cm]
\node[vertex2] (root){\nodepart{one} 54
					 \nodepart{two}  };
\vertexnodeShort{child1}{below left=1cm and 2cm of root}{34}{40}
\vertexnodeShort{child2}{below right=1cm and 2cm of root}{76}{}

\vertexnodeShort{gchild1}{below left=of child1}{12}{}
\vertexnodeShort{gchild2}{below =of child1}{38}{}
\vertexnodeShort{gchild3}{below right=of child1}{44}{46}
\vertexnodeShort{gchild4}{below left=of child2}{76}{}
\vertexnodeShort{gchild5}{below =of child2}{68}{}
\vertexnodeShort{gchild6}{below right=of child2}{86}{}

\draw[->, very thick] (root.south west) -- (child1);
\draw[->, very thick] (root.south) -- (child2);

\draw[->, very thick] (child1.south west) -- (gchild1);
\draw[->, very thick] (child1.south) -- (gchild2);
\draw[->, very thick] (child1.south east) -- (gchild3);

\draw[->, very thick] (child2.south west) -- (gchild4);
\draw[->, very thick] (child2.south) -- (gchild5);
\draw[->, very thick] (child2.south east) -- (gchild6);
\end{tikzpicture}

\textbf{38: L�schen mit Mischen}\\
\begin{tikzpicture}[node distance= 1cm and 1cm]
\tikzstyle{vertex} = [rectangle split, rectangle split horizontal, rectangle split parts=4, draw, text width = 0.3cm, align = center, minimum height=0.5cm, minimum width=2cm]
\tikzstyle{vertex2} = [rectangle split, rectangle split horizontal, rectangle split parts=2, draw, text width = 0.3cm, align = center, minimum height=0.5cm, minimum width=1cm]
\node[vertex2] (root){\nodepart{one} 54
					 \nodepart{two}  };
\vertexnodeShort{child1}{below left=1cm and 2cm of root}{40}{}
\vertexnodeShort{child2}{below right=1cm and 2cm of root}{76}{}

\vertexnodeShort{gchild1}{below left=of child1}{12}{34}
%\vertexnodeShort{gchild2}{below =of child1}{38}{}
\vertexnodeShort{gchild3}{below right=of child1}{44}{46}
\vertexnodeShort{gchild4}{below left=of child2}{76}{}
\vertexnodeShort{gchild5}{below =of child2}{68}{}
\vertexnodeShort{gchild6}{below right=of child2}{86}{}

\draw[->, very thick] (root.south west) -- (child1);
\draw[->, very thick] (root.south) -- (child2);

\draw[->, very thick] (child1.south west) -- (gchild1);
%\draw[->, very thick] (child1.south) -- (gchild2);
\draw[->, very thick] (child1.south) -- (gchild3);

\draw[->, very thick] (child2.south west) -- (gchild4);
\draw[->, very thick] (child2.south) -- (gchild5);
\draw[->, very thick] (child2.south east) -- (gchild6);
\end{tikzpicture}

\textbf{12, 44: Einfaches L�schen}\\
\begin{tikzpicture}[node distance= 1cm and 1cm]
\tikzstyle{vertex} = [rectangle split, rectangle split horizontal, rectangle split parts=4, draw, text width = 0.3cm, align = center, minimum height=0.5cm, minimum width=2cm]
\tikzstyle{vertex2} = [rectangle split, rectangle split horizontal, rectangle split parts=2, draw, text width = 0.3cm, align = center, minimum height=0.5cm, minimum width=1cm]
\node[vertex2] (root){\nodepart{one} 54
					 \nodepart{two}  };
\vertexnodeShort{child1}{below left=1cm and 2cm of root}{40}{}
\vertexnodeShort{child2}{below right=1cm and 2cm of root}{76}{}

\vertexnodeShort{gchild1}{below left=of child1}{34}{}
%\vertexnodeShort{gchild2}{below =of child1}{38}{}
\vertexnodeShort{gchild3}{below right=of child1}{46}{}
\vertexnodeShort{gchild4}{below left=of child2}{76}{}
\vertexnodeShort{gchild5}{below =of child2}{68}{}
\vertexnodeShort{gchild6}{below right=of child2}{86}{}

\draw[->, very thick] (root.south west) -- (child1);
\draw[->, very thick] (root.south) -- (child2);

\draw[->, very thick] (child1.south west) -- (gchild1);
%\draw[->, very thick] (child1.south) -- (gchild2);
\draw[->, very thick] (child1.south) -- (gchild3);

\draw[->, very thick] (child2.south west) -- (gchild4);
\draw[->, very thick] (child2.south) -- (gchild5);
\draw[->, very thick] (child2.south east) -- (gchild6);
\end{tikzpicture}
\section*{Aufgabe 3: Berechnungen in B- und B*-B�umen}
\subsection*{a) Max und Min des B*-Baumes}
i) 343*10 = 3430\\
ii) 50*5 = 250

\subsection*{b) B-Baum - Werte f�r $h$}
Auf allen Ebenen (ab der Wurzel) k�nnten alle Seiten / Knoten
h�chstens 8 Datens�tze haben.\\
Bei einer H�he von 2 passen alsoschon 8 + 9*8 = 80 Datens�tze hinein.\\
Die Wurzel muss minimal einen Datensatz besitzen und alle anderen Knoten mindestens 4.\\
Bei einer H�he von 3 w�ren dann noch nicht alle Datens�tze untergebracht:
1 + 2*4 + 10*4 = 49.\\
Also bewegt sich die H�he zwischen 2 und 4.

\subsection*{c) B*-Baum - Anzahl Datens�tze}
2*42 = 84

\section*{Aufgabe 4: Normalformenlehre}
\subsection*{i) Schl�sselkandidaten}
\begin{verbatim}
B    oder
B, A oder
B, C oder
A, C
\end{verbatim}
\subsection*{ii) Nicht-Prim�rattribute}
\begin{verbatim}
E, D
\end{verbatim}

\subsection*{iii) Normalform des Relationsschemas}
Wir nehmen an 'B' wird als Prim�rschl�ssel verwendet.\\ 
Das Relationsschema ist in 1NF und auch in 2NF, denn kein Nicht-Prim�rattribut ist partiell von einem Schl�sselattribut abh�ngig. Dies sieht man schon daran, dass 'B' atomar ist.\\
Das Relationsschema ist nicht in 3NF, da 'D' transitiv �ber 'A,C' von 'B' abh�ngt.


\end{document}